\documentclass{article}

% Language setting
% Replace `english' with e.g. `spanish' to change the document language
\usepackage[english]{babel}
\usepackage{ctex}

% Set page size and margins
% Replace `letterpaper' with `a4paper' for UK/EU standard size
\usepackage[letterpaper,top=2cm,bottom=2cm,left=3cm,right=3cm,marginparwidth=1.75cm]{geometry}

% Useful packages
\usepackage{amsmath}
\usepackage{graphicx}
\usepackage[colorlinks=true, allcolors=blue]{hyperref}

\title{那一年我二十七岁,在我一生的黄金时代}
\author{谭颂华}
\date{2026年1月14日}

\begin{document}
\maketitle

\section{一些杂话}

在二十岁初,心里似乎没那么多杂念,心里单纯地想着考研(22岁以下)、Linear DAR的理论推导和代码实现(22--25岁)。站在28岁往回看,觉得一切都那么珍贵,珍贵的不是一系列成果,而是那时候每日都专注于手头一项项工作,每日都有新收获。而之后似乎陷入一阵阵的迷茫时光,半年的实习(25岁)让研究断了一阵子;波士顿交流的一年(26--27岁)从头开始,转入函数数据分析,眼高手低的缺点使得看似解决了很多关键问题,但论文收尾太慢;27岁毕业年,找工作带来的焦虑感如影随形;即使找到不错的工作,还是会有隐隐的不配得感。

Prof. Xiao 教会我很多道理,其中记忆最深的是 ``毕业后的几年是科研最关键的几年,冲过去就一切都好了。'' 感恩 Prof. Xiao 的教导,23年感恩节在 Prof. Xiao 家吃烤鸡,但今年感恩节却因为回国后进度缓慢而不敢发圣诞祝福。好在生命中另一位贵人和我说过,``二十岁是试错的年纪'',朱老师也不断提醒我,过去的事情就不要管了,专注于手头上的工作。

因此,我想在这封信中,坦诚地说出自己过去做得不好的地方,现在的一些困境,和未来的一些希望。保持坦诚和反省是我最喜欢的特质,如果老师您发现我再有本文出现的问题,或者任何做得不好得地方,请直接指出。这封信同时也是写给我自己,希望我能常读常新,希望明年写总结的时候,更加自信,更加积极。

\section{2025年总结}

2025年最大的几个事件是``找到中科大的特任副教授工作(1月)''、``成功申请到合作项目(8月)''、``阶段性完成之前函数型分位数自回归项目(12月)''、``陆续开始做新论文''。后续将分为个人部分和论文部分分别阐述各个事件带来的一些思考。

\subsection{重要事件}

2025年最重要的事件是

\subsection{个人部分}

\begin{enumerate}
    \item 身份转变:名义上,我完成了从``学生''到``独立科研者''的转变。但实际上,因为我25年12月份才招了第一个学生,26年3月份才开始上课。所以之前实际上没有感受到任何压力,最近才逐渐觉得带学生的责任感以及考核的压力感。
    \item 专注力:
    \item 坚定力:
\end{enumerate}

\subsection{论文部分}

\begin{enumerate}
    \item 学术品位:我时常把``读博''视为打工,而把``独立科研''看作创业。需要提出自己的一套被人认可的方法论。
    \item 科研主题:
\end{enumerate}

\section{2026年展望}

\subsection{个人部分}

\begin{enumerate}
    \item 保持好奇:
    \item 注入活力:
    \item 每日记录:
\end{enumerate}

\subsection{论文部分}
\begin{enumerate}
    \item 论文数量:
    \item 论文质量:
\end{enumerate}

\end{document}
